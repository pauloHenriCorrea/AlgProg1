// Para dar vermelho, HIGH para vermelho, LOW para verde e azul
// Para dar verde, HIGH para verde, LOW para vermelho e azul
// Para dar amarelo, HIGH para verde e vermelho e LOW para azul

float duracao, distancia;

// Cores atuais dos leds
int estado_led1 = 1, estado_led2 = 1;

// Calcula o tempo que passa dentro da função tempos_alternados
int i;

int delay2;
int estadoBotao;
int ultimoEstadoBotao = LOW;

const int pinoBotao = A1;
bool c = false;

// Variáveis da frente
const int LED_verde_f = 2;
const int LED_azul_f = 3;
const int LED_vermelho_f = 4;
const int buzzer_f = 5;
const int Echo_f = 6;
const int Trig_f = 7;

// Variáveis de tras_t
const int LED_verde_t = 8;
const int LED_azul_t = 9;
const int LED_vermelho_t = 10;
const int buzzer_t = 11;
const int Echo_t = 12;
const int Trig_t = 13;

// Inicializa todos as variáves que vão ser usadas no arduíno
void setup() {
  Serial.begin(9600);

  pinMode(Trig_f, OUTPUT);
  digitalWrite(Trig_f, LOW);
  delayMicroseconds(2);

  // Inicializando as portas dos leds da frente
  pinMode(LED_vermelho_f, OUTPUT);
  pinMode(LED_azul_f, OUTPUT);
  pinMode(LED_verde_f, OUTPUT);
  pinMode(buzzer_f, OUTPUT);
  pinMode(Echo_f, INPUT);
  
  // Botão
  pinMode(pinoBotao, INPUT);

  // Inicializando as portas dos leds de tras
  pinMode(LED_vermelho_t, OUTPUT);
  pinMode(LED_azul_t, OUTPUT);
  pinMode(LED_verde_t, OUTPUT);
  pinMode(buzzer_t, OUTPUT);
  pinMode(Echo_t, INPUT);
  pinMode(Trig_t, OUTPUT);
  
  digitalWrite(Trig_t, LOW);
  delayMicroseconds(2);

  delay(1000);
}

// Receba as portas, a frequência e delay de ambos os buzzers
// Faz com que os buzzers tocam em tempo distintos
void tempos_alternados(
  int porta_buzzer1, 
  int porta_buzzer2, 
  int delay1, 
  int freq1, 
  int freq2
) {
  
  // Regula quantos toques o mais rapido vai dar antes do devagar
  delay2 = delay1 * 2;  
  
  // nesse caso é 2
  i = 0;
  
  while (i <= delay2) {
    if (i % delay1 == 0 && i != delay2) {
      tone(porta_buzzer1, freq1);
      delay(delay1);

      i += delay1;

      noTone(porta_buzzer1);
      delay(delay1);
    }

    if (i == delay2) {
      tone(porta_buzzer2, freq2);
      delay(delay1 - 50);

      i += delay1 - 50;

      noTone(porta_buzzer2);
    }
  }
}


void Led_Buzzer(
    int Led_vermelho, 
    int Led_azul, 
    int Led_verde, 
    int vermelho, 
    int azul, 
    int verde, 
    int buzzer, 
    int freq
) {
  // Os 3 primeiros digitalWrite verifica qual a cor de led deseja acender
  // 1
  digitalWrite(Led_vermelho, vermelho);

  // 2
  digitalWrite(Led_azul, azul);

  // 3
  digitalWrite(Led_verde, verde);

  if (estado_led1 == 0 && estado_led2 == -1) {
    tempos_alternados(buzzer_t, buzzer_f, 250, 490, 400);
  } else if (estado_led1 == -1 && estado_led2 == 0) {
    tempos_alternados(buzzer_f, buzzer_t, 250, 490, 400);
  } else if (estado_led1 == -1 && estado_led2 == -1) {
    tone(buzzer, freq);
    delay(200);
    noTone(buzzer);
    delay(90);
  } else if (estado_led1 == -1 && estado_led2 == 1) {
    tone(buzzer_f, freq);
    delay(250);
    noTone(buzzer_f);
    delay(250);
  } else if (estado_led1 == 1 && estado_led2 == -1) {
    tone(buzzer_t, freq);
    delay(250);
    noTone(buzzer_t);
    delay(250);
  } else if (estado_led1 == 0 && estado_led2 == 1) {
    tone(buzzer_f, freq);
    delay(250);
    noTone(buzzer_f);
    delay(1250);
  } else if (estado_led1 == 1 && estado_led2 == 0) {
    tone(buzzer_t, freq);
    delay(250);
    noTone(buzzer_t);
    delay(1250);
  } else if (estado_led1 == 0 && estado_led2 == 0) {
    tone(buzzer_t, freq);
    delay(250);
    noTone(buzzer_t);
    delay(750);
    tone(buzzer_f, freq);
    delay(250);
    noTone(buzzer_f);
    delay(750);
  }
}

void sensorDist (
    int LED_verde, 
    int LED_azul, 
    int LED_vermelho,
    int buzzer, 
    int Echo, 
    int Trig
    ) {
    
    // Manda o sinal e espera receber de volta por 10 microsegundos
    digitalWrite(Trig, HIGH);     
    delayMicroseconds(10);

    // Fecha a porta para receber sinal pois o tempo acabou
    digitalWrite(Trig, LOW);

    duracao = pulseIn(Echo, HIGH);
    distancia = duracao / 58;

    if (distancia <= 30) {
        // Acende o LED vermelho
        Led_Buzzer(LED_vermelho, LED_azul, LED_verde, 1, 0, 0, buzzer, 495);

        if (LED_verde == 2) {
            estado_led1 = -1;
        } else {
            estado_led2 = -1;
        }
    } else if (distancia <= 50) {
        // Acende o LED amarelo
        Led_Buzzer(LED_vermelho, LED_azul, LED_verde, 1, 0, 1, buzzer, 495);

        if (LED_verde == 2) {
            estado_led1 = 0;
        } else {
            estado_led2 = 0;
        }
    } else {
        // Acende o LED verde
        Led_Buzzer(LED_vermelho, LED_azul, LED_verde, 0, 0, 1, 99, 495);

        if (LED_verde == 2) {
            estado_led1 = 1;
        } else {
            estado_led2 = 1;
        }
    }
    delay(50);
}

void loop() {
  estadoBotao = digitalRead(pinoBotao);

  // Liga e desliga o circuito
  if (c == false) {
    while (estadoBotao == 1) {
      estadoBotao = digitalRead(pinoBotao);
      c = true;
    }
  } else {
    while (estadoBotao == 1) {
      estadoBotao = digitalRead(pinoBotao);
      c = false;
    }
  }

  // Verifica se o ciucuito está ligado ou desligado
  if (c == true) {
    sensorDist(2, 3, 4, 5, 6, 7);
    sensorDist(8, 9, 10, 11, 12, 13);
  } else {
    digitalWrite(LED_vermelho_f, LOW);
    digitalWrite(LED_azul_f, LOW);
    digitalWrite(LED_verde_f, LOW);
    digitalWrite(LED_vermelho_t, LOW);
    digitalWrite(LED_azul_t, LOW);
    digitalWrite(LED_verde_t, LOW);
  }
}